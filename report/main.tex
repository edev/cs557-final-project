\documentclass{article}
\usepackage{amsmath,amssymb,amsthm,enumitem} % Some standard math packages.
\usepackage{titling} % Enables \setlength{\droptitle}
\usepackage{parskip} % Cleaner paragraph display
\usepackage[margin=0.75in]{geometry} % Adjusts margins.
\usepackage[utf8]{inputenc} % Use UTF-8 input encoding instead of default ASCII.
\usepackage{fancyvrb} % Allows Verbatim sections with line numbers and such. Note the capital V.
\usepackage{xcolor} % for text color, e.g. \todo command
\usepackage{ragged2e} % For text alignment environments, e.g. \begin{center}
\usepackage[]{forest} % Draws trees.
\usepackage{enumitem} % Allows different list enumerations, like a.) b.) c.)
\usepackage{hyperref} % For \href and \uri
\usepackage[english]{babel}
\usepackage[
  backend=biber,
  style=numeric,
  sorting=ynt
]{biblatex} % Bibliography
\addbibresource{bibliography.bib}
% Set center alignment within tree nodes, which allows multi-line nodes.
\forestset{qtree/.style={for tree={parent anchor=south, 
           child anchor=north,align=center,inner sep=0pt}}}

\newcommand {\todo}[1] {{\textbf{\color{red}#1}}}
\newcommand {\mq}[1] {\text{`$#1$'}}
\newcommand {\unit}[1] {\ensuremath{\ \text{#1}}}
\newcommand {\code}[1] {\texttt{#1}}
\newcommand {\term}[1] {\textbf{#1}}

\newenvironment{typewriter}{\ttfamily}{\par}

\begin{document}

% Header
\begin{center}
    \Huge Reimplementing a Ruby on Rails Site in Yesod \\
    \large CS 557: Functional Languages @ Portland State University \\
    Dylan Laufenberg, winter 2019
\end{center}

\section{Introduction}

For the last seven years or so, I've been developing and expanding a Ruby on Rails-powered website for a Sacramento, CA soup kitchen named Sharing God's Bounty. I've had mixed feelings about my choice of Ruby on Rails over the last half-dozen years, so this final project provides an excellent opportunity to experiment with a significantly different Web technology stack than any other that I've used. I implement a narrow, vertical slice of functionality to experience and experiment with Haskell across many aspects of server-side code for the Web. I use the Yesod Web framework for my project, and I provide comparisons to my Ruby on Rails experience as appropriate throughout this report. I follow the quick start instructions \cite{yesodQuickstart}, which have provided me with Yesod 1.6.0.3.

Yesod's documentation comes primarily from the Yesod book \cite{ybk}, so a large portion of my work for this project has simply been reading through the book to understand how to utilize this highly sophisticated, mature framework. I don't attempt to introduce or define every concept in Yesod here. Instead, I highlight particularly noteworthy applications of Haskell to the domain of Web application development and discuss how I use Yesod to build out the features I develop for this final project. In particular, I assume the reader knows about Haskell features such as Template Haskell, QuasiQuotes, and language pragmas either through prior knowledge or by reading the Haskell section of the Yesod book \cite{ybkHaskell}.



\section{Resources and type-safe URIs} \label{resourcesAndTSURIs}

To represent \term{routes}, which are mappings from URIs to server-side code, Yesod uses what it calls resources. A \term{resource} is a \code{data} constructor that acts as the canonical name of the route in views \cite{ybk}. For instance, the default site scaffold includes a static route of the form:

\begin{Verbatim}
/static StaticR Static appStatic
\end{Verbatim}

This line specifies a \code{/static} route, a resource constructor named \code{StaticR} as the name of the route, and a subsite named \code{Static} whose function \code{appStatic} will handle requests on this route \cite{ybkRouting}. In order to create a link to a static file, e.g. an image served as part of a template, we invoke the \code{StaticR} constructor with an appropriate argument. The \code{Static} subsite generates ``static file identifiers`` at compile time \cite{ybkScaffolding}. When we insert a link using a resource constructor:

\begin{Verbatim}
<img ... src="@{StaticR img_logo_png}" ... />
\end{Verbatim}

Yesod generates a correct link to the static resource at compile-time. In this case, the user's browser receives:

\begin{Verbatim}
<img ... src="http://localhost:3000/static/img/logo.png?etag=GVVUzjtL" ... />
\end{Verbatim}

\subsection{Benefits of Resources}

There are two primary benefits to this approach, both of which leverage Haskell features from class (albeit at a very advanced level).

\paragraph{Check links at compile time} Yesod uses resources to generate correct links at compile-time, which allows it to check whether said links are valid and raise compile-time errors if any problems are detected. In this case, the Static subisite checks the identifier \code{img\_logo\_png} against all generated static file identifiers. If it finds a match, it inserts the correct link. If not, then GHC raises an error. For instance, if we mistakenly specify a JPG logo:

\begin{Verbatim}
<img ... src="@{StaticR img_logo_jpg}" ... />
\end{Verbatim}

GHC will detect the issue for us and (rather obtusely) notify us of the issue with a compile-time error:

\begin{Verbatim}
    • Variable not in scope: img_logo_jpg :: Route Static
    • Perhaps you meant ‘img_logo_png’ (imported from Import.NoFoundation)
    |
163 |             $(widgetFile "default-layout")
    |               ^^^^^^^^^^^^^^^^^^^^^^^^^^^
\end{Verbatim}

In effect, Yesod automatically tests every internal link at compile time, without the need to manually write tests against the generated HTML. We, the developers, need to test our functions for generating links, then Yesod takes care of the rest for us. In this case, I feel comfortable as a developer taking it on faith that the static file subsite is well-tested.

\paragraph{Refactor routes with confidence} Generating links at compile time also eliminates an entire class of common and nefarious errors: broken links. Updating or moving routes is a common occurrence in Web application development, even with careful planning. Yesod makes this safe and automatic by generating all such links from resource constructors at compile time. To change the route across the entire application, all we need to do is change the route definition:

\begin{Verbatim}
/dynamic StaticR Static appStatic
\end{Verbatim}

And all of our generated links to that route will automatically update when we recompile:

\begin{Verbatim}
<img ... src="http://localhost:3000/dynamic/img/logo.png?etag=GVVUzjtL" ... />
\end{Verbatim}

% This is also a great point of comparison to Ruby or Python, where this is impossible by the very nature of the languages.


\section{The basics}

I'm not sure what should go here, exactly, but the chapter on the basics definitely deserves some discussion AS IT PERTAINS TO MY CODE. I don't think there's much value in covering things I didn't write.

\section{Shakespearean Templates}

Topics:
\begin{itemize}
  \item Hamlet: went pretty smoothly, Initially had to cut out lots of features as NYI.
  \item Lucius/Cassius: oh, man. Whole subsection(s), whole big thing.
\end{itemize}

\section{Reflections after the basics}

Reflect on Yesod from the perspective of someone who's just made it through the basics. In particular, TinyOS's observation about being hard for beginners. I feel like I have to be an expert to do almost anything. Every piece of code feels like it's written for experts. And the lack of an accessible API reference (with good documentation) so far seems to be frustrating me.

% \input{p2.tex}
% \input{treeVisualizer.tex}
% \input{p3.1.tex}
% \input{p3.2.tex}

\printbibliography
\end{document}
