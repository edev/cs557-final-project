\section{Introduction}

For the last seven years or so, I've been developing and expanding a Ruby on Rails-powered website for a Sacramento, CA soup kitchen named Sharing God's Bounty. I've had mixed feelings about my choice of Ruby on Rails over the last half-dozen years, so this final project provides an excellent opportunity to experiment with a significantly different Web technology stack than any other that I've used. I implement a narrow, vertical slice of functionality to experience and experiment with Haskell across many aspects of server-side code for the Web. I use the Yesod Web framework for my project, and I provide comparisons to my Ruby on Rails experience as appropriate throughout this report. I follow the quick start instructions \cite{yesodQuickstart}, which have provided me with Yesod 1.6.0.3.

Yesod's documentation comes primarily from the Yesod book \cite{ybk}, so a large portion of my work for this project has simply been reading through the book to understand how to utilize this highly sophisticated, mature framework. I don't attempt to introduce or define every concept in Yesod here. Instead, I highlight particularly noteworthy applications of Haskell to the domain of Web application development and discuss how I use Yesod to build out the features I develop for this final project. In particular, I assume the reader knows about Haskell features such as Template Haskell, QuasiQuotes, and language pragmas either through prior knowledge or by reading the Haskell section of the Yesod book \cite{ybkHaskell}.

