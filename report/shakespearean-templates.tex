\section{Shakespearean Templates}

Yesod provides domain-specific languages (DSLs) over HTML, CSS, and Javascript. In converting my site, I migrated the HTML, CSS, and Javascript, in that order. (For the full, reference versions of all information in this section that's not otherwise cited, see \cite{ybkShakes}.)

\todo{Cover interpolation: variables, type-safe URLs, and templates}

\subsection{Hamlet}

Yesod uses \term{Hamlet} as its HTML processing DSL. Hamlet is similar to HTML, with a few distinguishing features:

\begin{itemize}
  \item \term{Significant whitespace} Hamlet templates infer closing tags from indentation levels. When you write, for instance, \code{<div>}, everything that should appear inside this tag must be nested below it. A matching closing tag is automatically inserted at the correct place in the generated HTML.
  \item \term{Class and ID shorthand} Hamlet templates provide shorthand for specifying classes and IDs by prefacing a class or ID name with a \code{.} or a \code{\#}, respectively.
  \item \todo{Anything else here?}
\end{itemize}

\todo{Move to reflections?} Unfortunately, through my own testing (and mistakes), I've found that Hamlet's value for checking the correctness of generated HTML is highly limited.

Topics:
\begin{itemize}
  \item Hamlet: went pretty smoothly, Initially had to cut out lots of features as NYI.
  \item Lucius/Cassius: oh, man. Whole subsection(s), whole big thing.
\end{itemize}

%   Migrate main HTML template, creating a list of features snipped out:
%     Christmas mode
%     Mobile menu
%     Meals served & last meal
%     Announcements
%     Main menu
%     Flash notice and alert
%     Footer: current year
%     Footer: login link
%   Migrate main CSS stylesheet to Lucius:
%     Integrate variables where appropriate
%     Convert image URLs to StaticR resources for compile-time safety
%     Nest CSS with shared tags (e.g. all .masthead selectors)
%     Mixins:
%       - Can't use the Yesod book's quasi-quoters readily in the scaffolded Yesod site, either in the Lucius file or in src/Foundation.hs (where defaultLayout is defined), because I need to annotate a type manually to deal with string overloading (Text vs. [char]), and I can't easily find the type of the QuasiQuoter used (because it far exceeds my level of knowledge of Haskell, GHCI, QuasiQuoters, and Yesod). 
%       - Google doesn't have any usable documentation on this that I've found.
%       - The Yesod wiki seems to be broken
%       - The LTS build of Yesod appears to be over 5 years old and appears to predate this: https://www.yesodweb.com/blog/2013/07/runtime-lucius-mixins
%       - Because of either Stack or Yesod's package management, I can't import Text.Lucius from GHCI to check the type.
%       - However, I eventually realized that I could trick Stack/Yesod into telling me the correct type simply by writing a dummy type and checking the 'Expected' output! Oputput:
%               • Couldn't match expected type ‘Mixin’ with actual type ‘Int’
%               • In the first argument of ‘Text.Internal.Css.CDMixin’, namely
%                   ‘(centering "200px")’
%                 In the expression: Text.Internal.Css.CDMixin (centering "200px")
%                 In the expression:
%                   ((Text.Shakespeare.Base.DerefBranch
%                       (Text.Shakespeare.Base.DerefIdent
%                          (Text.Shakespeare.Base.Ident "centering")))
%                      (Text.Shakespeare.Base.DerefString "200px"), 
%                    Text.Internal.Css.CDMixin (centering "200px"))
%               |
%           163 |             $(widgetFile "default-layout")
%               |               ^^^^^^^^^^^^^^^^^^^^^^^^^^^
% 
%           --  While building package final-project-0.0.0 using:
%                 /home/dylan/.stack/setup-exe-cache/x86_64-linux-tinfo6/Cabal-simple_mPHDZzAJ_2.2.0.1_ghc-8.4.4 --builddir=.stack-work/dist/x86_64-linux-tinfo6/Cabal-2.2.0.1 build lib:final-project --ghc-options " -ddump-hi -ddump-to-file"
%               Process exited with code: ExitFailure 1
%       - I changed the type annotation to Text -> Mixin and it worked perfectly.
%       - I learned while creating the HTML template that older versions of Yesod won't recognize new static files unless I stack clean && stack exec -- yesod devel, and that the LTS release I received from Stack is old enough that it has this issue.
%       - I learned while trying to get Mixins working that I also have to clean and rebuild if I change a Lucius mixin.
%     Limitation of variables:
%       - I tried converting my 20/10-pixel spacing to @spacing and @halfSpacing, with @halfSpacing defined as @halfSpacing = calc(#{spacing}/2);, but it doesn't process #{spacing} or even produce a compile error! It actually produces incorrect CSS!
%       - I found that if I leave a semicolon off a variable, it will also produce invalid CSS! What gives?!
%     Next steps:
%       1. Get good dummy content in
%       2. Get the CSS working well and looking right!
%       3. Continue refactoring the CSS
